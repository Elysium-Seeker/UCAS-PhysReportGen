% 本模板根据中国科学院大学本科生公共必修课程《基础物理实验》Word模板格式编写
% 本项目已被列为课程推荐使用模板. 
% 本模板由Shing-Ho Lin和Jun-Xiong Ji于2022年9月共同完成, 旨在方便LaTeX原教旨主义者和被Word迫害者写实验报告, 避免Word文档因插入过多图与公式造成卡顿. 
% 适配报告生成器自动化版本

\documentclass[11pt]{article}

\usepackage[a4paper]{geometry}
\geometry{left=2.0cm,right=2.0cm,top=2.5cm,bottom=2.5cm}

\usepackage{ctex} % 支持中文的LaTeX宏包
\usepackage{float} % 支持 [H] 强制位置
\usepackage{amsmath,amsfonts,graphicx,subfigure,amssymb,bm,amsthm,mathrsfs,mathtools,breqn} % 数学公式和符号的宏包集合
\usepackage{algorithm,algorithmicx} % 算法和伪代码的宏包
\usepackage[noend]{algpseudocode} % 算法和伪代码的宏包
\usepackage{fancyhdr} % 自定义页眉页脚的宏包
\usepackage[framemethod=TikZ]{mdframed} % 创建带边框的框架的宏包
\usepackage{fontspec} % 字体设置的宏包
\usepackage{adjustbox} % 调整盒子大小的宏包
\usepackage{fontsize} % 设置字体大小的宏包
\usepackage{tikz,xcolor} % 绘制图形和使用颜色的宏包
\usepackage{multicol} % 多栏排版的宏包
\usepackage{multirow} % 表格中合并单元格的宏包
\usepackage{pdfpages} % 插入PDF文件的宏包
\RequirePackage{listings} % 在文档中插入源代码的宏包
\RequirePackage{xcolor} % 定义和使用颜色的宏包
\usepackage{wrapfig} % 文字绕排图片的宏包
\usepackage{bigstrut,multirow,rotating} % 支持在表格中使用特殊命令的宏包
\usepackage{booktabs} % 创建美观的表格的宏包
\usepackage{circuitikz} % 绘制电路图的宏包

\definecolor{dkgreen}{rgb}{0,0.6,0}
\definecolor{gray}{rgb}{0.5,0.5,0.5}
\definecolor{mauve}{rgb}{0.58,0,0.82}
\lstset{
  frame=tb,
  aboveskip=3mm,
  belowskip=3mm,
  showstringspaces=false,
  columns=flexible,
  framerule=1pt,
  rulecolor=\color{gray!35},
  backgroundcolor=\color{gray!5},
  basicstyle={\small\ttfamily},
  numbers=none,
  numberstyle=\tiny\color{gray},
  keywordstyle=\color{blue},
  commentstyle=\color{dkgreen},
  stringstyle=\color{mauve},
  breaklines=true,
  breakatwhitespace=true,
  tabsize=3,
}

% 轻松引用, 可以用\cref{}指令直接引用, 自动加前缀. 
\usepackage[capitalize]{cleveref}
\Crefname{section}{Section}{Sections}
\Crefname{table}{Table}{Tables}
\crefname{table}{Table.}{Tabs.}

\setmainfont{Palatino Linotype.ttf}[Path=./]
\setCJKmainfont{SimSun.ttf}[
    Path=./,
    BoldFont=SimHei.ttf,
    ItalicFont=Songti.ttf
]
\setCJKsansfont{SimHei.ttf}[Path=./]
\setCJKmonofont{SimHei.ttf}[Path=./]
\newCJKfontfamily\kaishu{Songti.ttf}[Path=./] % 兼容模板中的 \kaishu 或 \emph
\punctstyle{kaiming}

\renewcommand{\emph}[1]{{\kaishu#1}}

%改这里可以修改实验报告表头的信息
\newcommand{\experiName}{实验名称}
\newcommand{\supervisor}{指导教师}
\newcommand{\name}{姓名}
\newcommand{\studentNum}{学号}
\newcommand{\class}{1}
\newcommand{\group}{01}
\newcommand{\seat}{1}
\newcommand{\dateYear}{2026}
\newcommand{\dateMonth}{1}
\newcommand{\dateDay}{1}
\newcommand{\room}{实验室}
\newcommand{\others}{$\square$}
%% 如果是调课、补课, 改为: $\square$\hspace{-1em}$\surd$
%% 否则, 请用: $\square$
%%%%%%%%%%%%%%%%%%%%%%%%%%%

\begin{document}

%若需在页眉部分加入内容, 可以在这里输入
% \pagestyle{fancy}
% \lhead{\kaishu 测试}
% \chead{}
% \rhead{}

\begin{center}
    \LARGE \bfseries 《\, 基\, 础\, 物\, 理\, 实\, 验\, 》\, 实\, 验\, 报\, 告
\end{center}

\begin{center}
    \noindent \emph{实验名称}\underline{\makebox[25em][c]{\experiName}}
    \emph{指导教师}\underline{\makebox[8em][c]{\supervisor}}\\
    \emph{姓名}\underline{\makebox[6em][c]{\name}} 
    % 如果名字比较长, 可以修改box的长度"6em"
    \emph{学号}\underline{\makebox[10em][c]{\studentNum}}
    \emph{分班分组及座号} \underline{\makebox[5em][c]{\class \ -\ \group \ -\ \seat }\emph{号}} (\emph{例}:\, 1\,-\,04\,-\,5\emph{号})\\
    \emph{实验日期} \underline{\makebox[3em][c]{\dateYear}}\emph{年}
    \underline{\makebox[2em][c]{\dateMonth}}\emph{月}
    \underline{\makebox[2em][c]{\dateDay}}\emph{日}
    \emph{实验地点}\underline{{\makebox[10em][c]\room}}
    \emph{调课/补课} \underline{\makebox[4em][c]{\others\ 是}}
    \emph{成绩评定} \underline{\hspace{5em}}
    {\noindent}
    \rule[8pt]{17cm}{0.2em}
\end{center}

% BEGIN:content
\section{实验目的}
\begin{enumerate}
    \item 请填写实验目的
\end{enumerate}

\section{实验器材}
请填写实验器材

\section{实验原理}
请填写实验原理

\section{实验步骤}
\begin{enumerate}
    \item 请填写实验步骤
\end{enumerate}

\section{实验结果与数据处理}
请填写实验结果与数据处理

\section{思考题}
\begin{enumerate}
    \item 思考题1
    
    答: 请填写答案
\end{enumerate}

\section{总结}
请填写实验总结
% END:content

% BEGIN:appendix
\section{附录}

\subsection{预习报告照片}
% 预习报告图片将在此插入

\subsection{实验原始数据记录表照片}
% 数据记录表图片将在此插入
% END:appendix

\end{document}
